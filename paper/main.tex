\documentclass[12pt,a4paper]{article}

\usepackage[T1]{fontenc}
\usepackage[russian]{babel}
\usepackage[utf8]{inputenc}
\usepackage{amsmath}
\usepackage{authblk}
\usepackage{extsizes}
\usepackage{graphicx}
\usepackage{mathptmx}
\usepackage{fullwidth}
\usepackage{svg}
\usepackage{amsmath}
\usepackage[backend=biber]{biblatex}
\usepackage{wrapfig}
\usepackage{geometry}

\geometry{
    a4paper,
    total={210mm,297mm},
    left=30mm,
    right=15mm,
    top=20mm,
    bottom=20mm,
}

\addbibresource{bibliography.bib}

\setlength{\parskip}{1em}

\title{Многоцелевой автономный малый беспилотный летательный аппарат}

\author{Кириенко П. С., Стойка Е. В., Кроль А. В.}

\affil{Zubax Robotics, г. Москва}

\date{}

\begin{document}

\maketitle

\section{Введение}

В настоящее время широкую популярность приобретают малые беспилотные летательные аппараты (МБПЛА) в различных прикладных задачах от аэрофотосъёмки до грузоперевозок. Мировой опыт эксплуатации различной авиации показывает, что безопасное и эффективное применение летательных аппаратов (ЛА) требует высокой квалификации оператора (пилота), и что именно оператор является наименее надёжным элементом в задаче пилотирования ЛА \cite{RiskManagementHandbookFAA, PlaneCrashInfo, HumanFactorsBoeing}. Кроме того, высокая стоимость рабочего времени квалифицированного оператора ограничивает множество задач, где использование МБПЛА показало бы экономическую эффективность \cite{DroneHire}.

В этой работе мы рассматриваем основные проблемы, связанные с разработкой адекватной системы управления МБПЛА, способной максимально исключить человека из процесса пилотирования; предлагаем архитектуру программно-аппаратного обеспечения, и приводим реализации некоторых основных алгоритмов, выработанных к настоящему моменту в ходе продолжающегося исследования.

Ключевые отличия нашего подхода от используемых на сегодняшний день в индустрии следующие:

\begin{itemize}
    \item Использование компьютерного зрения для автономного счисления позиции, скорости и ориентации МБПЛА. Это обеспечивает надёжную локализацию даже при полной недоступности спутниковой навигационной системы (СНС).
    \item Использование компьютерного зрения для построения модели окружающего пространства. Это даёт системе управления возможность безопасно пилотировать МБПЛА в сложных условиях без участия человека-оператора.
    \item Алгоритм комплексирования показаний различных сенсоров, способный корректно определять позицию, скорость и ориентацию БПЛА даже при недоступности измерений СНС и магнитного компаса.
\end{itemize}

Эта публикация состоит из 3-х основных частей. В первой части мы производим анализ задачи автоматизации процесса пилотирования МБПЛА с целью минимизации участия человека. Во второй части мы рассматриваем возможную архитектуру программно-аппаратного обеспечения такого МБПЛА. В третьей части мы приводим практическую реализацию двух ключевых компонентов такой системы: блока компьютерного зрения и надёжного алгоритма позиционирования.

\section{Анализ задачи}



\section{Архитектура МБПЛА}

\begin{figure}[!h]
    \centering
    \includesvg[width=1\textwidth]{Arch}
    \caption{\label{fig:arch}Архитектура программно-аппаратного обеспечения высокоавтономного МБПЛА.}
\end{figure}



\section{Система компьютерного зрения}



\section{Алгоритм позиционирования}

Предлагаемый алгоритм позиционирования основывается на расширенном фильтре Калмана и решает задачу надёжной оценки состояния МБПЛА, включая позицию, скорость и ориентацию в пространстве. Отличительными особенностями предлагемого алгоритма являются:

\begin{itemize}
    \item Использование оценки позиции и ориентации на основе компьютерного зрения, что позволяет предотвратить накопление ошибки даже при длительной недоступности СНС.
    \item Способность алгоритма к надёжной оценке ориентации по азимуту на основе измерений СНС, что позволяет избежать применения сенсоров, основанных на менее надёжных принципах измерений, таких как магнитные компасы или датчики солнца.
\end{itemize}

Дальнейшее описание использует следующие определения:

\begin{itemize}
    \item $g$ - стандартное ускорение свободного падения на поверхности Земли.
    \item $q^*$ - комплексное сопряжённое $q$.
    \item $a \otimes b$ - произведение кватернионов $a$ и $b$ \cite{QuaternionsMadgwick}.
\end{itemize}

\subsection{Модель сенсоров}

Модель сенсоров основывается на следующих предположениях \cite{weiss2012vision}:

\begin{itemize}
    \item ИНС находится в центре масс МБПЛА, система координат (СК) ИНС совпадает с СК МБПЛА.
    \item Каждый сенсор имеет собственную СК, относительно которой выполняются измерения, и трансформация из локальной СК каждого сенсора в СК ИНС известна и постоянна.
    \item Измерения позиции, полученные от системы компьютерного зрения, выполнены в СК камеры.
    \item Измерения позиции, полученные от системы компьютерного зрения, подвержены медленному накоплению ошибки позиции и ориентации по мере движения МБПЛА.
\end{itemize}

Визуализация используемой модели систем координат приведена на иллюстрации \ref{fig:frames}, где:

\begin{itemize}
    \item $p_i^g$ - смещение антенны приёмника СНС относительно ИНС. Следует обратить внимание, что модель не предусматривает вращения антенны, т.к. этот параметр не влияет на наблюдаемые измерения этого сенсора.
    \item $p_i^c$ $q_i^c$ - смещение и поворот СК камеры относительно ИНС. Согласно широко принятому соглашению \cite{ROSFrames}, ось $z_c$ камеры совпадает с её оптической осью, ось $y_c$ направлена вниз, ось $x_c$ направлена вправо.
    \item $p_v^c$ $q_v^c$ - позиция и ориентация, согласно измерениям системы компьютерного зрения. Как было замечено выше, эти измерения подвержены медленному накоплению ошибки позиции и ориентации.
    \item $p_v^w$ $q_v^w$ - накопленная ошибка (смещение) оценки позиции и ориентации системой компьютерного зрения.
    \item $p_i^w$ $q_i^w$ - позиция и ориентация ИНС относительно глобальной СК.
\end{itemize}

\begin{figure}[!h]
    \centering
    \includesvg[width=0.7\textwidth]{frames}
    \caption{\label{fig:frames}Модель систем координат (СК).}
\end{figure}

\subsection{Вектор состояния}

Компоненты вектора состояния приведены в формуле \ref{eq:x} и детально рассмотрены ниже.

\begin{equation}
    \label{eq:x}
    x=\left(
    \begin{array}{c}
    p_w^i \\
    v_w^i \\
    q_w^i \\
    a \\
    w \\
    j_a \\
    j_w \\
    b_a \\
    b_w \\
    p_v^w \\
    q_v^w \\
    \end{array}
    \right)
\end{equation}

Компоненты вектора состояния:
\begin{itemize}
    \item $p_w^i$ - См. рис. \ref{fig:frames}.
    \item $v_w^i$ - Скорость ИНС относительно глобальной СК.
    \item $q_w^i$ - См. рис. \ref{fig:frames}.
    \item $a$ - Ускорение ИНС в ИСО.
    \item $w$ - Угловая скорость ИНС в ИСО.
    \item $j_a$ - Рывок ИНС в ИСО (первая производная ускорения) \cite{Kishore94}.
    \item $j_w$ - Угловое ускорение ИНС в ИСО \cite{Kishore94}.
    \item $b_a$ - Дрейф акселерометра ИНС.
    \item $b_w$ - Дрейф ДУС ИНС.
    \item $p_v^w$ - См. рис. \ref{fig:frames}.
    \item $q_v^w$ - См. рис. \ref{fig:frames}.
\end{itemize}

Вращения в векторе состояния представлены в виде кватернионов поворота \cite{QuaternionsMadgwick}, линейные перемещения представлены в виде трёхмерных векторов, что в итоге порождает вектор состояния из 35 вещественных переменных. Состояние фильтра включает в себя производные позиции и ориентации высокого порядка, поскольку было показано преимущество подобного подхода в задачах определения состояния высокодинамических систем \cite{Kishore94}. Ошибка оценки состояния представлена согласно классическому подходу, в виде ковариационной матрицы $35 \times 35$.

Функция предсказания состояния $f(x,\text{$\Delta $t})$, где $\Delta t$ - интервал предсказания, приведена ниже.

\begin{equation}
    \label{eq:f}
    f(x,\text{$\Delta $t})=\left(
    \begin{array}{c}
    p_w^i+\text{$\Delta $t} v_w^i+\text{$\Delta $t}^2 \text{qrot}\left(a+\text{$\Delta $t} j_a,\left(q_w^i\right){}^*\right) \\
    v_w^i+\text{$\Delta $t} \text{ } \text{qrot}\left(a+\text{$\Delta $t} j_a,\left(q_w^i\right){}^*\right) \\
    q_w^i\otimes \text{quaternion}\left(j_w \text{$\Delta $t}^2+w \text{$\Delta $t}\right) \\
    a+\text{$\Delta $t} j_a \\
    w+\text{$\Delta $t} j_w \\
    j_a \\
    j_w \\
    b_a \\
    b_w \\
    p_v^w \\
    q_v^w \\
    \end{array}
    \right)
\end{equation}

Определение некоторых тривиальных функций, используемых в модели фильтра, приведено ниже \cite{QuaternionsMadgwick, QuaternionsNASA}.

\begin{equation}
    \label{eq:quaternion}
    \text{quaternion}(\phi ,\theta ,\psi )=\left(
    \begin{array}{c}
    w \\
    x \\
    y \\
    z \\
    \end{array}
    \right)=\left(
    \begin{array}{c}
    \cos \left(\frac{\theta }{2}\right) \cos \left(\frac{\phi }{2}\right) \cos \left(\frac{\psi }{2}\right)+\sin \left(\frac{\theta }{2}\right) \sin \left(\frac{\phi }{2}\right) \sin \left(\frac{\psi }{2}\right) \\
    \cos \left(\frac{\theta }{2}\right) \cos \left(\frac{\psi }{2}\right) \sin \left(\frac{\phi }{2}\right)-\cos \left(\frac{\phi }{2}\right) \sin \left(\frac{\theta }{2}\right) \sin \left(\frac{\psi }{2}\right) \\
    \cos \left(\frac{\phi }{2}\right) \cos \left(\frac{\psi }{2}\right) \sin \left(\frac{\theta }{2}\right)+\cos \left(\frac{\theta }{2}\right) \sin \left(\frac{\phi }{2}\right) \sin \left(\frac{\psi }{2}\right) \\
    \cos \left(\frac{\theta }{2}\right) \cos \left(\frac{\phi }{2}\right) \sin \left(\frac{\psi }{2}\right)-\cos \left(\frac{\psi }{2}\right) \sin \left(\frac{\theta }{2}\right) \sin \left(\frac{\phi }{2}\right) \\
    \end{array}
    \right)
\end{equation}

\begin{equation}
    \label{eq:euler}
    \text{euler}(w,x,y,z)=\left(
    \begin{array}{c}
    \tan ^{-1}\left(1-2 \left(x^2+y^2\right),2 (w x+y z)\right) \\
    \sin ^{-1}(2 (w y-x z)) \\
    \tan ^{-1}\left(1-2 \left(y^2+z^2\right),2 (x y+w z)\right) \\
    \end{array}
    \right)
\end{equation}

\begin{equation}
    \label{eq:qrot}
    \text{qrot}(\text{vec},q)=q\otimes \text{vec}\otimes q^*
\end{equation}

\subsection{Модель измерений}

В этой части приводятся уравнения предсказания измерений расширенного фильтра Калмана.

Функции предсказания измерений акселерометра и ДУС ИНС:

\begin{equation}
    h_{\text{acc}}(x)=\text{qrot}\left(\left(
    \begin{array}{c}
    0 \\
    0 \\
    g \\
    \end{array}
    \right),q_w^i\right)+b_a+a
\end{equation}

\begin{equation}
    h_{\text{gyro}}(x)=b_w+w1
\end{equation}

Функции предсказания измерений позиции и скорости СНС:

\begin{equation}
    h_{\text{gnsspos}}(x)=p_w^i
\end{equation}

\begin{equation}
    h_{\text{gnssvel}}(x)=v_w^i
\end{equation}

Функции предсказания измерений позиции, скорости и ориентации, полученные от системы компьютерного зрения:

\begin{equation}
    h_{\text{vispos}}(x)=\text{qrot}\left(p_v^w,\left(q_v^w\right){}^*\right)+p_w^i
\end{equation}

\begin{equation}
    h_{\text{visvel}}(x)=\text{qrot}\left(v_w^i,\left(q_v^w\right){}^*\right)
\end{equation}

\begin{equation}
    h_{\text{visatt}}(x)=\text{euler}\left(q_w^i\otimes q_v^w\right)P
\end{equation}

\subsection{Реализация}

Описанный выше расширенный фильтр Калмана реализован на языке C++ для системы Robotic Operating System (ROS) \cite{ROS} с использованием разработанного в рамках данного исследования инструмента синтеза исходного кода по высокоуровневому описанию уравнений фильтра. Процесс синтеза происходит следующим образом:

\begin{enumerate}
    \item Уравнения фильтра описываются на языке Wolfram Mathematica.
    \item Полученное описание используется для генерации исходного текста C++ класса вектора состояния фильтра, для чего применяется разработанная в рамках данного исследования утилита.
    \item Сгенерированнй класс вектора состояния используется C++ приложением, где реализуются низкоуровневые задачи обработки измерений и обновления состояния фильтра.
\end{enumerate}

Исходные тексты описанных компонентов можно найти по URL: \url{https://github.com/Zubax/zubax_posekf}.

\subsection{Результаты}

Тестирование фильтра выполнялось с использованием свободно доступных тестовых данных CAMPUS-0L и CAMPUS-2L \cite{blanco2009cor}.

(...)

\section{Заключение}



\nocite{*}
\newpage
\begin{fullwidth}
\printbibliography
\end{fullwidth}

\end{document}
